\documentclass[fleqn, 10pt]{article}

%Definimos el título
\title{Teoria de Autómatas y Lenguajes Formales \\[.4\baselineskip] Práctica 1: Latex y expresiones regurales}
\author{Piedad, Paredes Garcia}
\date{\today}

%Comienzo del documento
\begin{document}

%Generamos el título
\maketitle

EJERCICIO 1
\\[.8\baselineskip]
Para $ n=1: \mathcal{R} ^1 = \mathcal{R}$ \\
Para $n\geq2: (a,b) \in \mathcal{R} ^n $ sii $ \exists x \in A | (a,x) \in \mathcal{R} ^{n-1}$ and $(a,b) \in \mathcal{R}$ \\[.8\baselineskip]
$\mathcal{R} ^1 = \{(1,1),(1,2),(2,3),(3,4)\}$\\
$\mathcal{R} ^2 = \{(1,1),(1,2),(1,3),(2,4)\}$\\
Una vez calculados $\mathcal{R} ^1$ y $\mathcal{R} ^2$ podemos calcular $\mathcal{R} ^3 $\\
 $\mathcal{R} ^3 = \{(1,1),(1,2),(1,3),(1,4)\}$ \\[.8\baselineskip]
 
 Ahora comprobamos la solucion abriendo powerrelation.m en Octave e insertando en la ventana de comandos: \\
 powerrelation({['1', '1'], ['1', '2'], ['2', '3'],['3', '4']}, 3)
 
 



\end{document}

